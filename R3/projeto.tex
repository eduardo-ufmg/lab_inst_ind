\documentclass{article}
\usepackage{graphicx} % Required for inserting images
\usepackage{array} % provides >{...} and \arraybackslash for column formatting
\usepackage[utf8]{inputenc}
\usepackage[brazil]{babel}
\usepackage{amsmath}
\usepackage{geometry}
\usepackage{longtable}
\usepackage{float}

\usepackage{microtype} % better line breaking and character protrusion
\usepackage{grffile}   % allow graphics filenames with spaces/multiple dots
\usepackage[hyphens]{url} % allow breaks at hyphens in URLs/paths
\usepackage{seqsplit}  % allow breaking very long unbreakable strings

% Relax typesetting to avoid overfull boxes (use with care)
\emergencystretch=1.5em
\tolerance=800
\hyphenpenalty=200
\sloppy

\geometry{a4paper, margin=1in}

\title{Relatório 3: Correção e Modernização da Documentação da planta STEC e Calibração do Sensor de Nível}
\author{Eduardo Henrique Basilio de Carvalho \\ Renan Neves da Silva}
\date{Dezembro de 2025}

\begin{document}

\maketitle

\section{Introdução}

O presente relatório tem como objetivo apresentar as atividades desenvolvidas para a modernização da documentação e configuração da planta didática STEC (Sistema de Tanques para Estudo de Controle). 

O trabalho abrange a revisão do Fluxograma de Processo e Instrumentação (P\&ID) para adequação à norma ANSI/ISA-5.1-2009, estabelecendo uma nova nomenclatura padronizada e a divisão da planta em unidades lógicas. Além disso, descreve-se a estratégia de configuração da rede Foundation Fieldbus e o procedimento de calibração do transmissor de nível (LIT-101), incluindo a análise de incertezas e erros de medição, visando garantir a conformidade normativa e a confiabilidade metrológica do sistema.

\section{Documentação e Projeto}

\subsection{Nova Estrutura de Identificação}

A planta foi dividida em duas unidades principais para a numeração dos instrumentos: 

\begin{itemize}
    \item Unidade 100: Tanque de Aquecimento (TAQ) e linhas associadas;
    \item Unidade 200: Tanque de Produto (TP) e linhas associadas.
\end{itemize}

As letras de As letras de identificação foram atualizadas para remover os sufixos "CV" dos TAGs de válvulas, passando a usar apenas "V", conforme a norma ISA 5.1-2009.

\subsection{Lista de Instrumentos}

Nesta tabela, os instrumentos são listados com suas novas identificações:

\begin{longtable}{>{\raggedright\arraybackslash}p{2cm} >{\raggedright\arraybackslash}p{2.2cm} >{\raggedright\arraybackslash}p{4cm} >{\raggedright\arraybackslash}p{2cm} >{\raggedright\arraybackslash}p{2.2cm}}
\caption{Lista de Instrumentos — novas identificações}
\label{tab:lista-instrumentos} \\
\hline
\textbf{Novo TAG} & \textbf{TAG Antigo} & \textbf{Descrição} & \textbf{Unidade} & \textbf{Tipo de Sinal} \\
\hline
\endfirsthead

\multicolumn{5}{c}{{\tablename\ \thetable{} -- continuação}} \\
\hline
\textbf{Novo TAG} & \textbf{TAG Antigo} & \textbf{Descrição} & \textbf{Unidade} & \textbf{Tipo de Sinal} \\
\hline
\endhead

\hline \multicolumn{5}{r}{{Continua na próxima página}} \\ \hline
\endfoot

\hline
\endlastfoot

\multicolumn{5}{l}{\textbf{Malha 101 - Controle de Nível do TAQ}} \\
LIT-101 & LIT1 & Transmissor Indicador de Nível & TAQ & Fieldbus \\
LIC-101 & LC1 & Controlador Indicador de Nível & TAQ & Fieldbus \\
LY-101 & - & Conversor de Sinal & Campo & 4-20 mA \\
LV-101 & LCV1 & Válvula de Controle de Nível & TAQ & Pneumático \\
LSH-101 & - & Chave de Nível Alto & TAQ & Fieldbus \\
LSL-101 & - & Chave de Nível Baixo & TAQ & Fieldbus \\
\hline
\multicolumn{5}{l}{\textbf{Malha 102 - Controle de Temperatura do TAQ}} \\
TIT-102 & TIT1 & Transmissor Indicador de Temperatura & TAQ & Fieldbus \\
TIC-102 & TC1 & Controlador Indicador de Temperatura & TAQ & Fieldbus \\
TY-102A & - & Conversor de Sinal & Campo & 4-20 mA \\
TY-102B & - & Conversor de Sinal & Campo & Tensão \\
TY-102C & - & Módulo de Potência & TAQ & Elétrico de Potência \\
TZ-102 & Resistor & Elemento Final de Temperatura & TAQ & Elétrico de Potência \\
TE-102 & TE1 & Elemento Sensor de Temperatura & TAQ & Físico \\
\hline
\multicolumn{5}{l}{\textbf{Malha 201 - Controle de Nível do TP}} \\
LIT-201 & LIT2 & Transmissor Indicador de Nível & TP & Fieldbus \\
LIC-201 & LC2 & Controlador Indicador de Nível & TP & Fieldbus \\
LY-201 & - & Conversor de Sinal & Campo & 4-20 mA \\
LV-201 & LCV2 & Válvula de Controle de Nível & TP & Pneumático \\
LSH-201 & - & Chave de Nível Alto & TP & Fieldbus \\
LSL-201 & - & Chave de Nível Baixo & TP & Fieldbus \\
\hline
\multicolumn{5}{l}{\textbf{Malha 202 - Controle de Temperatura do TP}} \\
TIT-202 & TIT2 & Transmissor Indicador de Temperatura & TP & Fieldbus \\
TIC-202 & TC2 & Controlador Indicador de Temperatura & TP & Fieldbus \\
TY-202A & - & Conversor de Sinal & Campo & 4-20 mA \\
TE-202 & TE2 & Elemento Sensor de Temperatura & TP & Físico \\
TY-202B & - & Conversor de Sinal & Campo & 4-20 mA \\
TV-202 & TCV1 & Válvula de Controle de Temperatura & TP & Pneumático \\
\hline
\multicolumn{5}{l}{\textbf{Malha 203 - Controle de Vazão de Saída do TP}} \\
FIT-203 & FIT4 & Transmissor Indicador de Vazão & TP & Fieldbus \\
FIC-203 & FC4 & Controlador Indicador de Vazão & TP & Fieldbus \\
FY-203 & - & Conversor de Sinal & Campo & 4-20 mA \\
FV-203 & FCV1 & Válvula de Controle de Vazão & TP & Pneumático \\
\hline
\multicolumn{5}{l}{\textbf{Instrumentos de Indicação e Outros}} \\
FIT-103 & FIT1 & Transmissor de Vazão (Linha de entrada TAQ) & TAQ & Fieldbus \\
FIT-204 & FIT3 & Transmissor de Vazão (Linha de entrada TP Fria) & TP & Fieldbus \\
FIT-205 & FIT2 & Transmissor de Vazão (Linha de entrada TP Quente) & TP & Fieldbus \\
LG-101 & LG1 & Visor de Nível (Local) & TAQ & Visual \\
LG-201 & LG2 & Visor de Nível (Local) & TP & Visual \\
P-101 & BA1 & Bomba Centrífuga (Alimentação) & TR & Elétrico \\
P-201 & BA2 & Bomba Centrífuga (Recirculação/Drenagem) & TP & Elétrico \\
KM-101 & - & Contator de alimentação da Bomba P-101 & TR & Elétrico de Potência \\
KM-201 & - & Contator de alimentação da Bomba P-201 & TP & Elétrico de Potência \\
KM-102 & - & Contator de alimentação de TZ-102 & TAQ & Elétrico de Potência \\
\multicolumn{5}{l}{\textbf{CLP, Rede e Computadores}} \\
CLP-1 & CLP1 & Controlador Lógico Programável & Campo & Serial \\
FFGW-1 & - & Gateway Fieldbus Foundation & Campo & Fieldbus \\
PC-1 & PC1 & Computador de Supervisão & Sala de Controle & Serial / Fieldbus \\
\end{longtable}

\subsection{Descritivo Funcional das Malhas}

\subsubsection{Malha de Controle de Nível do TAQ}
A finalidade desta malha é manter o nível de água do Tanque de Aquecimento (TAQ) no setpoint definido. O transmissor de nível LIT-101 (pressão diferencial) comunica a variável de processo ao controlador LIC-101 via rede Fieldbus. O controlador opera em malha fechada, calcula a ação corretiva e envia o comando ao conversor LY-101, que converte o sinal para o formato adequado ao atuador e pilota a válvula pneumática LV-101, modulando a entrada de água proveniente do reservatório. Medidas de segurança e alarmes locais estão associadas a limites alto/baixo de nível.

\subsubsection{Malha de Controle de Temperatura do TAQ}
Esta malha regula a temperatura do fluido no TAQ. O transmissor TIT-102 mede a temperatura e transmite ao controlador TIC-102. O controlador compara com o setpoint e transmite o sinal de controle para o módulo de potência TY-102, que atua sobre o elemento final TZ-102 (resistor de potência), que opera em controle proporcional modulado para ajustar a potência de aquecimento e manter a temperatura desejada. Sinalização e supervisão são realizadas via Fieldbus.

\subsubsection{Malha de Controle de Nível do TP}
Responsável por manter o nível do Tanque de Produto (TP) no valor de setpoint. O transmissor LIT-201 mede o nível e envia ao controlador LIC-201. O controlador atua sobre a válvula LV-201, regulando a entrada de água fria (diretamente do reservatório) para o TP. O laço é projetado para resposta estável à variação de carga, com alarmes configurados para condições de nível crítico.

\subsubsection{Malha de Controle de Temperatura do TP}
Controla a temperatura da mistura no Tanque de Produto. A medida de temperatura é feita por TIT-202 e enviada a TIC-202, que calcula a ação de controle. O atuador é a válvula de controle TV-202, que regula o fluxo de água quente proveniente do TAQ; ao modular esse fluxo, ajusta-se a energia fornecida ao TP e, consequentemente, sua temperatura.

\subsubsection{Malha de Controle de Vazão de Saída}
Esta malha controla a vazão de escoamento do TP para manter a taxa de saída desejada. O transmissor de vazão FIT-203 fornece a medida ao controlador FIC-203, que atua sobre a válvula FV-203 (linha de saída), permitindo direcionar o fluxo para retorno ao reservatório ou descarte. O controlador mantém o setpoint de vazão e garante estabilidade diante de perturbações de nível e entrada.

\subsection{P\&ID}

A figura \ref{fig:peid} mostra o diagrama de projeto atualizado.

\begin{figure}[H]
    \centering
    \includegraphics[width=\textwidth]{peid.png}
    \caption{Diagrama P\&ID atualizado da planta STEC}
    \label{fig:peid}
\end{figure}

\section{Configuração da Rede Foundation Fieldbus}

A configuração do sistema de automação foi realizada utilizando o software Syscon (System302 da SMAR). A arquitetura de rede implementada segue a topologia mista definida no projeto, composta por:

\begin{itemize}
    \item \textbf{Nível HSE (High Speed Ethernet):} Rede de alta velocidade (100 Mbps) para supervisão e controle, interligando a estação de engenharia e a controladora DFI302.
    \item \textbf{Bridge (DFI302):} Atua como gateway entre as redes HSE e H1, além de processar blocos lógicos complexos.
    \item \textbf{Nível H1 (31.25 kbps):} Barramento de campo onde residem os instrumentos inteligentes.
\end{itemize}

A estratégia de controle foi distribuída nos dispositivos de campo. Para cada malha (ex: Malha 202 - Temperatura do TP), instanciaram-se os blocos funcionais conforme a norma:
\begin{enumerate}
    \item \textbf{Bloco AI (Analog Input):} No transmissor TIT-202, responsável pela aquisição e linearização da temperatura.
    \item \textbf{Bloco PID:} Configurado para executar o algoritmo de controle, recebendo a PV do bloco AI e o Setpoint (SP) via supervisório.
    \item \textbf{Bloco AO (Analog Output):} No conversor FI302 (TAG CONV-02), recebendo o sinal de controle (MV) do PID e convertendo-o em corrente (4-20 mA) para atuar na válvula pneumática TV-202.
\end{enumerate}

Todos os dispositivos foram configurados com os blocos mandatórios \textit{Resource} (RSC), \textit{Transducer} (TRD) e \textit{Display} (DSP) para garantir o correto gerenciamento de recursos e diagnóstico.

\ref{fig:scada} mostra a tela inicial do supervisório do sistema.

\begin{figure}[H]
    \centering
    \includegraphics[width=0.5\textwidth]{dados e enunciado/prints da aula/SCADA.PNG}
    \caption{Supervisório da planta}
    \label{fig:scada}
\end{figure}

A captura \ref{fig:scada_malhas} mostra as telas de controle das malhas implementadas.

\begin{figure}[H]
    \centering
    \includegraphics[width=0.5\textwidth]{dados e enunciado/prints da aula/SCADA2.PNG}
    \caption{Telas de controle das malhas}
    \label{fig:scada_malhas}
\end{figure}


Os blocos configuração dos instrumentos de temperatura e vazão é apresentada na figura \ref{fig:syscon_temp_vazao}.

\begin{figure}[H]
    \centering
    \includegraphics[width=0.5\textwidth]{dados e enunciado/prints da aula/Syscon1.PNG}
    \caption{Configuração dos instrumentos de temperatura e vazão no Syscon}
    \label{fig:syscon_temp_vazao}
\end{figure}

A figura \ref{fig:syscon_nivel_conv} mostra a configuração do transmissor de nível e dos conversores de sinal no software Syscon.

\begin{figure}[H]
    \centering
    \includegraphics[width=0.5\textwidth]{dados e enunciado/prints da aula/Syscon2.PNG}
    \caption{Configuração do transmissor de nível e conversores de sinal no Syscon}
    \label{fig:syscon_nivel_conv}
\end{figure}

Por fim, a figura \ref{fig:syscon_clp} mostra a configuração do gateway CLP no software Syscon.

\begin{figure}[H]
    \centering
    \includegraphics[width=0.5\textwidth]{dados e enunciado/prints da aula/Syscon3.PNG}
    \caption{Configuração do gateway CLP no Syscon}
    \label{fig:syscon_clp}
\end{figure}

\section{Calibração do Transmissor de Nível LIT-101}

O procedimento de calibração do transmissor de nível LIT-101 foi realizado comparando-se os valores indicados pelo instrumento (via rede Fieldbus) com os valores de referência observados no visor de nível local (padrão), cobrindo a faixa de operação do Tanque de Aquecimento (TAQ).

Considerando as características construtivas da planta, foram observadas as seguintes elevações para a análise da pressão da coluna líquida:
\begin{itemize}
    \item Altura entre o mínimo do visor e a saída de água ($h_1$): 65 mm.
    \item Altura entre a tomada de pressão e o sensor ($h_2$): 265 mm.
    \item Altura entre o máximo do visor e o ladrão ($h_3$): 20 mm.
\end{itemize}

\subsection{Dados Coletados}

A Tabela \ref{tab:dados_calibracao} apresenta os dados coletados durante o ensaio.

\begin{table}[H]
    \centering
    \caption{Dados de Calibração: Visor Local vs. LIT-101}
    \label{tab:dados_calibracao}
    \begin{tabular}{|c|c|c|}
        \hline
        \textbf{Ponto} & \textbf{Padrão - Visor (mm)} & \textbf{Indicado - LIT-101 (mm)} \\
        \hline
        1  & 485.0 & 480.5 \\
        2  & 449.0 & 448.7 \\
        3  & 398.0 & 395.3 \\
        4  & 350.0 & 347.7 \\
        5  & 298.0 & 297.9 \\
        6  & 252.0 & 252.2 \\
        7  & 201.0 & 202.0 \\
        8  & 147.0 & 149.8 \\
        9  & 104.0 & 105.8 \\
        10 & 65.0  & 66.8  \\
        \hline
    \end{tabular}
\end{table}

\subsection{Curva de Calibração e Regressão Linear}

A partir dos dados experimentais, aplicou-se o método dos mínimos quadrados para obter a curva de calibração estática, relacionando a indicação do instrumento ($y$) com o valor padrão ($x$). A equação da reta ajustada é dada por:

\begin{equation}
    y = 0.9863 \cdot x + 3.5435
\end{equation}

O coeficiente de determinação obtido foi $R^2 = 0.99994$, indicando um altíssimo grau de linearidade e ajuste do modelo aos dados. A Figura \ref{fig:calibracao_lit101} ilustra a curva de calibração e os pontos medidos.

\begin{figure}[H]
    \centering
    \includegraphics[width=0.8\textwidth]{calibracao_lit101.png}
    \caption{Curva de Calibração do LIT-101 com reta de regressão ajustada}
    \label{fig:calibracao_lit101}
\end{figure}

\subsection{Análise de Incertezas e Erros}

A incerteza do sistema de medição foi estimada considerando a dispersão dos dados em relação à reta ajustada (incerteza do tipo A). Conforme metodologia simplificada para esta etapa, desconsiderou-se a incerteza herdada do padrão e a covariância dos parâmetros da reta.

\begin{itemize}
    \item \textbf{Desvio Padrão dos Resíduos ($S_y$):} 1.2093 mm.
    \item \textbf{Incerteza Padrão ($u$):} Calculada pela razão entre $S_y$ e o coeficiente angular ($a$), resultando em $1.2260$ mm.
    \item \textbf{Graus de Liberdade ($\nu$):} Com $n=10$ pontos, $\nu = n - 2 = 8$.
    \item \textbf{Incerteza Expandida ($U$):} Para um nível de confiança de 95,45\%, utilizou-se o fator de abrangência t-Student $t_{95.45, 8} \approx 2.37$.
\end{itemize}

\begin{equation}
    U = t \times u = 2.3664 \times 1.2260 \approx 2.90 \text{ mm}
\end{equation}

Portanto, o resultado da calibração indica uma incerteza expandida de \textbf{2.90 mm} para o transmissor.

\subsubsection{Linearidade e Erro Fiducial}

A linearidade foi calculada como o maior resíduo absoluto em relação à reta de calibração, e o erro fiducial como o maior erro absoluto em relação ao padrão dividido pelo fundo de escala (Span = 500 mm).

\begin{itemize}
    \item \textbf{Linearidade:} 2.32 mm (0.46\% do Span).
    \item \textbf{Erro Máximo (Leitura - Padrão):} 4.50 mm.
    \item \textbf{Erro Fiducial:} 0.90\%.
\end{itemize}

Os resultados demonstram que o transmissor opera dentro de limites aceitáveis de precisão para o controle de nível proposto na planta STEC.

\section{Conclusão}

O trabalho cumpriu os objetivos de modernização da documentação técnica e garantia da confiabilidade metrológica da planta didática STEC. As atividades desenvolvidas permitiram a integração entre os conceitos normativos de instrumentação e a prática de configuração de redes industriais.

No que tange à documentação, a revisão do Fluxograma de Processo e Instrumentação (P\&ID) e a reestruturação das tags conforme a norma ANSI/ISA-5.1-2009 proporcionaram uma identificação mais lógica e rastreável dos ativos, dividindo a planta nas unidades de Aquecimento (100) e Produto (200). Essa padronização é fundamental para a manutenção e operação segura do sistema.

A configuração da rede Foundation Fieldbus via software Syscon validou a arquitetura de controle distribuído. A correta instanciação e encadeamento dos blocos funcionais (AI, PID, AO) e dos blocos de recursos (Resource e Transducer) permitiram que a estratégia de controle fosse efetivamente implementada nos dispositivos de campo, garantindo a comunicação entre a instrumentação e o sistema supervisório.

Em relação à calibração do transmissor de nível LIT-101, a análise estatística dos dados coletados comprovou a linearidade do instrumento, evidenciada por um coeficiente de determinação ($R^2$) próximo da unidade. A incerteza expandida calculada de aproximadamente 2.90 mm, para um nível de confiança de 95.45\%, demonstra que o erro de medição é pouco significativo em relação à faixa de operação do tanque (Span de 500 mm). A análise de erros fiduciais confirmou que o sensor opera em limites aceitáveis de precisão.

Conclui-se, portanto, que a planta STEC encontra-se apta para operações de controle, com sua instrumentação devidamente documentada, configurada e validada metrologicamente. O procedimento destacou a importância da calibração periódica e da análise de incertezas para garantir a qualidade das variáveis de processo em malhas de controle industrial.

\end{document}
